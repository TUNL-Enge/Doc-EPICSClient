% Created 2015-08-12 Wed 12:04
\documentclass[11pt]{article}
\usepackage[utf8]{inputenc}
\usepackage[T1]{fontenc}
\usepackage{fixltx2e}
\usepackage{graphicx}
\usepackage{longtable}
\usepackage{float}
\usepackage{wrapfig}
\usepackage{rotating}
\usepackage[normalem]{ulem}
\usepackage{amsmath}
\usepackage{textcomp}
\usepackage{marvosym}
\usepackage{wasysym}
\usepackage{amssymb}
\usepackage{hyperref}
\tolerance=1000
\usepackage{fullpage}
\usepackage{times}
\date{2015-07-20}
\title{EPICS Client Install}
\hypersetup{
  pdfkeywords={},
  pdfsubject={},
  pdfcreator={Emacs 24.3.1 (Org mode 8.2.4)}}
\begin{document}

\maketitle

\section{Install notes}
\label{sec-1}
This assumes you have a clean install of Linux Mint (I'm using 17.1 KDE)

\subsection{Preparation}
\label{sec-1-1}
\begin{itemize}
\item Install a few things
\begin{verbatim}
sudo apt-get install ssh emacs build-essential libreadline-dev \
		     mesa-common-dev libgl1-mesa-dev python-dev
\end{verbatim}
\end{itemize}
\subsection{EPICS Base}
\label{sec-1-2}
\begin{itemize}
\item Boot into Mint
\item Log in as user
\item Make a new user: enge, if you don't have it
\begin{verbatim}
sudo adduser --home /home/enge enge
\end{verbatim}
\item Add that user to the sudoers list
\begin{verbatim}
sudo adduser enge sudo
\end{verbatim}
\item Log out and then in as that user
\item Make the file structure (I actually put these in a separate
partition and linked to them from here)
\begin{verbatim}
mkdir bin
mkdir project
mkdir GUI
\end{verbatim}
\item Download and untar the EPICS base package
\begin{verbatim}
wget http://www.aps.anl.gov/epics/download/base/base-3.15.2.tar.gz 
tar -xzvf base-3.15.2.tar.gz
ln -s base-3.15.2 base
\end{verbatim}
\item Paste the following into .bashrc
\begin{verbatim}
######################################################################
##  EPICS
######################################################################

## Base
export EPICS_ROOT=/home/enge
export EPICS_BASE=${EPICS_ROOT}/base/
export EPICS_HOST_ARCH=`${EPICS_BASE}/startup/EpicsHostArch`
export EPICS_BASE_BIN=${EPICS_BASE}/bin/${EPICS_HOST_ARCH}
export EPICS_BASE_LIB=${EPICS_BASE}/lib/${EPICS_HOST_ARCH}
if [ "" = "${LD_LIBRARY_PATH}" ]; then
    export LD_LIBRARY_PATH=${EPICS_BASE_LIB}
else
    export LD_LIBRARY_PATH=${EPICS_BASE_LIB}:${LD_LIBRARY_PATH}
fi
export PATH=${PATH}:${EPICS_BASE_BIN}

## EPICS Extensions
export EPICS_EXT=${EPICS_ROOT}/extensions
export EPICS_EXT_BIN=${EPICS_EXT}/bin/${EPICS_HOST_ARCH}
export EPICS_EXT_LIB=${EPICS_EXT}/lib/${EPICS_HOST_ARCH}
if [ "" = "${LD_LIBRARY_PATH}" ]; then
    export LD_LIBRARY_PATH=${EPICS_EXT_LIB}
else
    export LD_LIBRARY_PATH=${LD_LIBRARY_PATH}:${EPICS_BASE_LIB}
fi
export EPICS_SYNAPPS_BASE=${EPICS_ROOT}/synApps
export EPICS_SYNAPPS_BIN=${EPICS_SYNAPPS_BASE}/support/utils
export PATH=${PATH}:${EPICS_EXT_BIN}:${EPICS_SYNAPPS_BIN}
\end{verbatim}
\item Load it
\begin{verbatim}
source ~/.bashrc
\end{verbatim}
\item Compile EPICS
\begin{verbatim}
cd base
make -j2
\end{verbatim}
\item Buy and drink some coffee
\item Once finished, check that it works
\begin{verbatim}
softIoc
\end{verbatim}
\item Epics should have started. Now run the IOC
\begin{verbatim}
iocInit
\end{verbatim}
\item You should see something that looks like
\begin{verbatim}
epics> iocInit  
Starting iocInit
############################################################################
## EPICS R3.15.2 $Date: Thu 2015-05-14 14:09:28 +0200$
## EPICS Base built Jul 17 2015
############################################################################
iocRun: All initialization complete
epics> exit
\end{verbatim}
\end{itemize}
\subsection{synApps (needed for streamdevice and serial connections)}
\label{sec-1-3}
\begin{itemize}
\item Get the extensions and msi first
\begin{verbatim}
cd ~
wget http://www.aps.anl.gov/epics/download/extensions/extensionsTop_20120904.tar.gz
tar -xzvf extensionsTop_20120904.tar.gz
wget http://www.aps.anl.gov/epics/download/extensions/msi1-7.tar.gz
cd extensions/src
tar -xzvf ../../msi1-7.tar.gz
cd msi1-7
make
\end{verbatim}
\item Install re2c (I don't know what it's for)
\begin{verbatim}
sudo apt-get install re2c
\end{verbatim}
\item Download and unzip synApps
\begin{verbatim}
cd ~
wget http://www.aps.anl.gov/bcda/synApps/tar/synApps_5_8.tar.gz
tar -xzvf synApps_5_8.tar.gz
ln -s synApps_5_8 synApps
\end{verbatim}
\item We don't need all the junk included
\begin{verbatim}
cd synApps/support/configure
emacs RELEASE
\end{verbatim}
\item Edit the SUPPORT line
\begin{verbatim}
SUPPORT=/home/enge/synApps/support
\end{verbatim}
\item Edit EPICS$_{\text{BASE}}$
\begin{verbatim}
EPICS_BASE=/home/enge/base
\end{verbatim}
\item Comment out (with a '\verb~#~') the modules we don't want
\begin{itemize}
\item \verb~ALLEN_BRADLEY~
\item \verb~AREA_DETECTOR~
\item \verb~ADCORE~
\item \verb~ADBINARIES~
\item \verb~CAPUTRECORDER~
\item \verb~CAMAC~
\item \verb~DAC128V~
\item \verb~DXP~
\item \verb~IP~
\item \verb~IP330~
\item \verb~IPUNIDIG~
\item \verb~OPTICS~
\item \verb~QUADEM~
\item \verb~SOFTGLUE~
\item \verb~VME~
\end{itemize}
\item Prepare the makefile
\begin{verbatim}
cd ~/synApps/support
make release
\end{verbatim}
\item Compile!
\begin{verbatim}
make -j2 rebuild
\end{verbatim}
\end{itemize}
\subsection{Tidy up}
\label{sec-1-4}
\begin{itemize}
\item Make a folder to keep zip files
\begin{verbatim}
cd ~
mkdir Downloads
mv *.tar.gz Downloads
\end{verbatim}
\end{itemize}
\subsection{Qt GUI stuff}
\label{sec-1-5}
I've quite liked using Qt as a GUI. So far, \texttt{EpicsQt} has worked
quite nicely, but I haven't tried to do anything complicated
yet. In the mean time, we should also install \href{http://epics.web.psi.ch/software/caqtdm/}{CaQtDM}.

\begin{enumerate}
\item Qt Install
\label{sec-1-5-1}
\begin{itemize}
\item Download Qt (includes Qt Creator) from the \href{http://www.qt.io/}{official website}
\item Make sure you look for the open source one
\item This should have saved a file
      \verb~qt-unified-linux-x64-2.0.2-1-online.run~ in my case.
\item Make a folder to put this in
\begin{verbatim}
sudo mkdir /opt/Qt
\end{verbatim}
\item Install Qt in the folder you just made
\begin{verbatim}
chmod 755 qt-unified-linux-x64-2.0.2-1-online.run
./qt-unified-linux-x64-2.0.2-1-online.run
\end{verbatim}
\item This should install Qt. Check
\begin{verbatim}
qtcreator &
\end{verbatim}
\item Now add the following in .bashrc
\begin{verbatim}
#### Qt
export PATH=/opt/Qt/5.5/gcc_64/bin:/opt/Qt/Tools/QtCreator/bin:${PATH}
export QWT_ROOT=/usr/local/qwt-6.1.2
export QWT_INCLUDE_PATH=/usr/local/qwt-6.1.2/include/
export LD_LIBRARY_PATH=/usr/local/qwt-6.1.2/lib/:/opt/Qt/5.5/gcc_64/lib:${LD_LIBRARY_PATH}
\end{verbatim}
\item Also install QWT
\item Download from \url{http://qwt.sourceforge.net/}
\begin{verbatim}
source ~/.bashrc
cd ~/Downloads
tar -xjvf qwt-6.1.2.tar.bz2
cd qwt-6.1.2
qmake
make
sudo make install
\end{verbatim}
\end{itemize}
\item CaQtDM Install
\label{sec-1-5-2}
\begin{itemize}
\item \url{https://github.com/caqtdm/caqtdm/archive/V3.9.4.tar.gz}
\item Download:
\begin{verbatim}
cd ~/GUI
wget https://github.com/caqtdm/caqtdm/archive/V3.9.4.tar.gz
tar -xzvf V3.9.4.tar.gz
mv V3.9.4.tar.gz ~/Downloads/caQtDM_V3.9.4.tar.gz
\end{verbatim}
\item caQtDM doesn't find variables on its own, so make sure
      \verb~caQtDM_Env~ has the right variables
\begin{verbatim}
if [ -z "$QTHOME" ];           then export   QTHOME=/opt/Qt;
fi
if [ -z "$QWTHOME" ];          then export   QWTHOME=/usr/local/qwt-6.1.2;
fi
if [ -z "$QWTINCLUDE" ];       then export   QWTINCLUDE=${QWTHOME}/include;
fi
if [ -z "$QWTLIB" ];           then export   QWTLIB=${QWTHOME}/lib;
fi
if [ -z "$EPICS_BASE" ];       then export   EPICS_BASE=/home/enge/base;
fi
if [ -z "$EPICSINCLUDE" ];     then export   EPICSINCLUDE=${EPICS_BASE}/include;
fi
if [ -z "$EPICSLIB" ];         then  export  EPICSLIB=${EPICS_BASE}/lib/$EPICS_HOST_ARCH;
fi
if [ -z "$EPICSEXTENSIONS" ];  then  export  EPICSEXTENSIONS=/home/enge/extensions;
fi
if [ -z "$QTCONTROLS_LIBS" ];  then export  QTCONTROLS_LIBS=`pwd`/caQtDM_Binaries;
fi
if [ -z "$CAQTDM_COLLECT" ];  then export  CAQTDM_COLLECT=`pwd`/caQtDM_Binaries;
fi
\end{verbatim}
\item Make sure python is defined as the correct version (I had to put
2.7) in \verb~caQtDM_Env~
\item Fix compilerSpecific.h
\begin{verbatim}
ln -s /home/enge/base/include/compiler/gcc/compilerSpecific.h /home/enge/base/include/
\end{verbatim}
\item Run the build script
\begin{verbatim}
./caQtDM_BuildAll
\end{verbatim}
\end{itemize}
\item EpicsQt Install
\label{sec-1-5-3}
\begin{itemize}
\item Download from \texttt{www.sourceforge.net/project/epicsqt} (I got version 3.1.0)
\item Extract
\begin{verbatim}
mv epicsqt-3.1.0-src.tar.gz ~/GUI
cd ~/GUI
tar -xzvf epicsqt-3.1.0-src.tar.gz
mv 3.1.0 EpicsQt-3.1.0
\end{verbatim}
\item Add some things to \verb~.bashrc~
\begin{verbatim}
## QtEpics
export QE_EPICS_BASE=${EPICS_BASE}
export EPICSQT_ROOT=${EPICS_ROOT}/GUI/EpicsQt-3.1.0
export EPICSCAQTDM_ROOT=${EPICS_ROOT}/GUI/caqtdm-3.9.4
export PATH=${PATH}:${EPICSQT_ROOT}/applications/QEGuiApp/bin:${EPICSCAQTDM_ROOT}/caQtDM_Binaries
export LD_LIBRARY_PATH=${LD_LIBRARY_PATH}:${EPICSQT_ROOT}/framework/designer:${EPICS_BASE}/lib/${EPICS}
export QT_PLUGIN_PATH=${EPICSQT_ROOT}/framework:${EPICSCAQTDM_ROOT}/caQtDM_Binaries
\end{verbatim}
\item and source: \verb,source ~/.bashrc,
\item For some reason, I found it easiest to do the rest of this
compilation using Qt Creator. So load that now
\begin{verbatim}
qtcreator &
\end{verbatim}
\item Make sure the correct version of Qt is being used. On a fresh
install this should be easy enough, but you'll need to be
careful if there are multiple versions of Qt on your computer.
\item Load the \verb~epicsqt.pro~ file in the EpicsQt base directory
\item Uncheck "shadow build" in "Projects"
\item Add multi-processor building if you like by adding '-j2' to the
make arguments
\item Hit the "build" button!\\
      There will be lots of warnings but eventually it will
finish. Hopefully without any errors\ldots{}
\item Close and reopen Qt Creator (from the command line)
\item Open a test GUI and make sure it works
\begin{itemize}
\item Open a form
\item Tools -> Form Editor -> About Qt Designer Plugins
\item Scroll down to make sure the EpicsQt plugins are loaded
\end{itemize}
\end{itemize}
\end{enumerate}
% Emacs 24.3.1 (Org mode 8.2.4)
\end{document}